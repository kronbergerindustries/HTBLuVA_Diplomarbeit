{
	\newcommand{\tabitem}{~~\llap{\textbullet}~~}
	\newenvironment{mytable}[1][{|X[1,c,m]|X[2.1,l,m]|}]{
		\begin{tabu} to \textwidth {#1}
			\hline
		}{
			
		\end{tabu}
	}
	
	\tabulinesep = 2.5mm
	\let\cleardoublepage\clearpage
	
\begin{center}
\begin{huge}
	\textbf{DIPLOMARBEIT}
\end{huge}

\begin{large}
	DOKUMENTATION
\end{large}
\end{center}
	
\begin{hyphenrules}{nohyphenation}
	\begin{center}
		\begin{mytable}
			Namen der Verfasser &
			Martin Kronberger, Jakob Lackner,\newline Simon Kern, Tobias Schmeisser \\
			\hline
			Jahrgang / Schuljahr &
			$5$AHET, $2020/2021$\\
			\hline
			Thema der Diplomarbeit & Entwicklung eines emissionsfreien Sportmotorrades\\
			\hline
			Kooperationspartner &
			Schrack for Students, Sigmatek , EBG GmbH, \\
			\hline
			Aufgabenstellung & Es soll ein vollständig elektrifiziertes E–Motorrad aus einen alten Model mit Verbrennungsmotor entwickelt werden. Der vorherige Verbrennungsmotor wird durch eine elektrische Steuer- und Motoreinheit ersetzt. Ebenfalls soll das Zweirad über ein eigenständiges Akkusystem, mit einem für das Motorrad individuellem Ladesystem verfügen. Über eine zentrale Steuereinheit wird die Kommunikation zwischen den Systemen und dem Menschen gewährleistet.\\
			\hline
			Realisierung & Als Chassis wird eine ausgeschlachtete Duacati Monster S4 2001 verwendet und mit einem eigensentwickelten Rahmen versehen. Angetrieben wird das Motorrad  über eine bürstenlose Synchronmaschine, welche über ein BMS gesteuertes 50,4 Volt Lithiumionen Akkupack versorgt wird. Das Moment wird vom Motor über ein Kettengetriebe mit eine Übersetzung von ungefähr 1/9 auf die Straße übertragen. Die Peripherie wird über einen Raspberry Pi Minicomputer und Taster am Lenker gesteuert. Ebenso wird über ihn die Benutzeroberfläche gesteuert, welche über ein 11,6 Zoll Touch Panel angezeigt und gesteuert werden kann.\\
			\hline
		\end{mytable}
		
		\begin{mytable}
			Ergebnisse & Zum Abgabezeitpunkt befindet sich die Komponenten des Motorrads im Entwicklungszustand. Der Rahmen ist noch in der Endphase der Fertigung. Der Motor und dessen Software sind einsatzbereit und müsste nur mehr eingebaut werden. Die Versorgung ist fertig entwickelt und muss ebenso nur mehr gefertigt werden. Die Steuerung der Benutzeroberfläche befindet sich im Prototypen-Status, beinhaltet jedoch noch einige experimentelle Funktionen, welche noch etwas Programmier- und Testzeit benötigen. Das gesamte Konzept ist vollständig und kann mit etwas mehr Zeit und Sponsorengelder fertiggestellt werden.\\
		\end{mytable}\vskip-0.42cm
		\begin{mytable}
			Möglichkeit der Einsichtnahme in die Arbeit & Die Diplomarbeit ist in gebundener Form sowohl in der Schulbibliothek als auch bei AV Prof. Dipl-Ing. (FH) Roland Holzer einzusehen. Darüber hinaus besitzt jedes Mitglied des Projektteams eine vollständige Version in gebundener und digitaler Form.\\
		\end{mytable}\vskip-0.42cm
		\begin{mytable}[{|X[0.995,c]|X[1,m]|X[1,m]|}]
			Approbation \newline (Datum / Unterschrift) &
			\hbox{\footnotesize{Prüfer:}} &
			\hbox{\footnotesize{Abteilungsvorstand:}} \\
			\hline
		\end{mytable}
	\end{center}
\end{hyphenrules}	
	
\newpage
	
\begin{center}
\begin{huge}
	\textbf{DIPLOMA THESIS}
\end{huge}

\begin{large}
	DOCUMENTATION
\end{large}
\end{center}

\begin{hyphenrules}{nohyphenation}
	\begin{center}
		\begin{mytable}
			Authors &
			Martin Kronberger, Jakob Lackner,\newline Simon Kern, Tobias Schmeisser \\
			\hline
			Form / Academic year &
			$5$AHET, $2020/2021$\\
			\hline
			Topic & Development of an emission-free sports motorcycle\\
			\hline
			Co-operation partners &
			Schrack for Students, Sigmatek , EBG GmbH, \\
			\hline
			Assignment of Tasks & The plan is to develop a fully electrified e-motorcycle. The previous internal combustion engine is replaced by an control unit and electric motor unit. The two-wheeler should also have an independent battery system with a charging system that is individual for the motorcycle. Communication between the systems and humans is ensured via a central control unit.\\
			\hline
			Realisation & A cannibalized Duacati Monster S4 2001 is used as the chassis and provided with a specially developed frame. The motorcycle is driven by a brushless synchronous machine, which is supplied by a BMS controlled 50.4 volt lithium ion battery pack. The torque is transmitted from the engine to the road via a chain gear with a ratio of  about 1/9. The periphery is controlled via a Raspberry Pi minicomputer and button on the handlebar. It also controls the user interface, which can be displayed and controlled via an 11.6 inch touch panel.\\
			\hline
		\end{mytable}
		
		\begin{mytable}
			Results & The components of the motorcycle are at the time of delivery in the state of development. The framework is still in its final stages Production. The engine and its software are ready for use and would just have to be built in. The supply is over developed and just needs to be manufactured. The Control of the user interface is in the prototype status, however still contains some experimental functions which are still unavailable\\
		\end{mytable}\vskip-0.42cm
		\begin{mytable}
			Accessibility of Diploma Thesis & The diploma thesis is available in the school library and at Prof. Dipl.-Ing (FH) Roland Holzer's office. Furthermore, each member of the project team has a complete version.\\
		\end{mytable}\vskip-0.42cm
		\begin{mytable}[{|X[0.995,c]|X[1,m]|X[1,m]|}]
			Approval \newline (Date/Signature) &
			\hbox{\footnotesize{Examiner:}} &
			\hbox{\footnotesize{Head of Department:}} \\
			\hline
		\end{mytable}
	\end{center}
\end{hyphenrules}

