\chapter{Akku und Ladekonzept}
\fancyfoot[C]{Kern}
\section{Section}

%% Übersicht %%%%%%%%%%%%%%%%%%%%%%%%%%%%%%%%%%%%%%%%%%%%%%%
\section{Übersicht}


\subsection{Aufgaben der Energieversorgung}

\subsection{Aufgaben des Batteriemanagement}

\subsection{Allgemeine Übersicht der Elektroinstallation}

%% Batteriemanagement %%%%%%%%%%%%%%%%%%%%%%%%%%%%%%%%%%%%%%%%%
\section{Batteriemanagement}

\subsection{Akkumulatoren}
\subsubsection{Lithiumbatterien}
Die geplanten Funktionen des Antriebssystems lassen sich grob in zwei Grundfunktionen einteilen.

\begin{itemize}
	\item Der Antrieb - Translation ist eine Grundfunktion eines jeden Verkehrsmittels
	\\ Durch die Umwandlung der elektrischen in kinetische Energie erfährt 
	\\ das gesamte System eine Translation.
	\item Die Steuereinheit - Steuerung und Kommunikation mit anderen Betriebsmitteln
	\\ Realisiert durch In- und Outputs, Datenübetragung mithilfe des CAN-Buses 
\end{itemize}

Nun Unterscheiden wir zwischen dem Hardware- und dem Softwareaufbau des Antriebssystems.

\newpage