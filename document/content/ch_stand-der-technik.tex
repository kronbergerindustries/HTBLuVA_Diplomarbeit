\chapter{Stand der Technik}

\section{Synchronmaschine mit Dauermagneterregung}
\subsection{Auswertung der Anrtiebswelle}

\section{Curtis Controller}

\subsection{Allgemeines}
\subsection{VCL}
\subsection{Feldorientierte Regelung}

\section{Leonard-Umformer}

\subsection{Allgemeines}
\subsection{ }

\section{KPI-Regler}

\subsection{Allgemeines}
\subsection{ }

\section{Steuereinheiten}

\section{Bussysteme}

\section{Akkusysteme}
Verschiedene Speicher für elektrische Energie, die auf einer elektrochemischen Basis basieren, nennt man Batterien oder auch Akkumulatoren. Elektrochemische Speicher haben in den vergangenen Jahren immer mehr an Bedeutung gewonnen und werden auch in Zukunft immer öfter Gebrauch finden. Möglich wurde unsere heutige elektronische Mobilität erst mit der Erfindung der galvanischen Zelle. Seit dieser Erfindung,die mit Hilfe eines Stromkreises chemische Energie in elektrische Energie umzuwandelt, hat sich über die jahrzentelange Weiterentwicklung der Batterie einiges getan. Die Einsatzmöglichkeiten von Akkumulatoren sind extrem vielfältig. Kleine Lithium-Ionen Akkus werden zum Beispiel als Knopfzellen in Smartphones verwendet. Jedoch können sie auch bis hin zu großen stationären Energiespeichern für erneuerbare Energien benutzt werden. Wie bereits vorher erwähnt, sind elektrische Energiespeicher ein wichtiger Bestandteil für den Erfolg der Elektromobilität geworden. Es gibt unzählig viele verschiedene Arten von Batterien, die sich im chemischen Aufbau, ihrer Form und natürlich in ihren Einsatzmöglichkeiten unterscheiden. Durch die äußerst besonderen chemischen Eigenschaften und die vielseitigen Anwendungsbereiche, hat sich der Lithium-Ionen Akku durchgesetzt.
\subsection{Batteriearten}

\subsubsection{Bleiakkumulator}
Die ersten Versuche, einen auf Blei basierenden Akkumulator zu entwickeln, wurden am Anfang des 19. Jahrhunderts durchgeführt. Industriell wurde der Bleiakku interessant, als Forscher und Chemiker zusammen 1880 ein Verfahren entwickelten, bei dem der Bleiakkumulator bereits nach wenigen Ladezyklen, eine hohe Kapazität erreichte. Der erste technisch einsetzbare Bleiakkumulator wurde 1886 von Henri Tudor entwickelt. Dieser besitzt eine Zellspannung von ungefähr 2V (abhängig vom Ladezustand), was eine durchaus große Spannung für sogenannte "wässrige Syteme" ist. Der Ausdruck "wässrige Systeme" leitet sich von dem Elektrolyt ab. Bei Bleiakkumulatoren wird wässrige Schwefelsäure als Elektrolyt verwendet. Im entladenen Zusatnd bestehen beide Pole aus Blei(II)-sulfat (PbSO4). Weiters besteht die Kathode aus Blei und Anode aus Bleioxid. Bleiakkumulatoren sollten keinesfalls tiefenentladen werden, da dies zu Schäden führt und den Akku unbrauchbar macht. Ein extrem großer Nachteil ist das Gewicht, da nur 30 bis 40Wh/kg erreicht werden können.
Diese Art von Akku zeichnet sich durch das kurzzeitige Zulassen hoher Ströme aus die zum Beispiel für Fahrzeug -bzw. Starterbatterien notwendig sind. Unter anderem sind 50 Prozent des Batteriemarkets von Bleiakkumulatoren belegt. Wie vorher bereits erwähnt werden dies oftmals in Autos, LKWs oder auch Motorräder verbaut.

\subsubsection{Nickel-Metallhybrid Akkumulatoren}
Die technischen Grundlagen des Nickel-Metallhybrid Akkumulator wurden von Stanford R. Ovshinsky und Masahiko Oshitani ab 1962 bis 1982 zur marktreifen Zelle entwickelt. Seit dem Jahr 2006 sind NiMH-Akkumulatoren mit geringer Selbstentladung auf dem Markt, die sich gegenüber herkömmlichen NiMH-Akkus durch eine deutlich reduzierte Selbstentladung auszeichnen. Die positive Elektrode eines Nickel-Metallhybrid Akkumulatoren(NiMH) besteht aus Nickel(II)-hydroxid wogegen die negative Elektrode aus einem Metallhybrid zusammensetzt. Als Elektorlyt verwendet dieser Akkumulator eine Wasserstoffspeicherlegierung aus Nickel und seltenen Erden. NiMH-Akkus erreichen bis zu 80Wh/kg. Sie sind vielfach in den üblichen Bauformen von Standardbatterien verbreitet und liefer pro Zelle eine Spannung von 1,2V. Oftmals werden sie als wiederaufladbare Alternative der gängigen Alkalibatterien in haushaltsüblichen Geräten eingesetzt. Ein großer Vorteil gegenüber den Nickel-Cadmium Batterien ist es, das der NiMH Akku nicht aus giftigen Cadmium besteht und er außerdem eine höher Energiedichte aufweist.
Der Anwendungsbereich von NiMH Akkumalatoren ist sehr vielfältig. Vorzugsweise kommen sie wie NiCd Akkus überall dort zur Anwendung, wo ein hoher Energiebedarf besteht und hohe Batteriekosten vermeiden werden sollten. Tyische Anwendungsbereich sind zum Beispiel Foto- Videogeräte, Elektroautos, Elektrowerkzeuge und noch viele mehr. NiMH Akkus werden außerdem oft als Energiespeicher für Notbeleutungsanlagen verwendet.

\subsubsection{Nickel-Cadmium Akkumulatoren}
1899 wurde der Nickel-Cadmium Akku von dem Schweden W. Jungner entwickelt. NiCd Akkus zeichnen sich dadurch aus, dass sie einen eingebauten Ent- und Überladeschutz integriert haben. Das hat zur Folge, dass man keine aufwendige elektronische Beschaltung durchführen muss. Als Material für die Kathode dieses Akkus verwendet man Nickeloxidhydroxid. Die Anode dagegen besteht aus dem giftigen Material Cadmium, welches jedoch eine äußerst hohe spezifische Ladung (478Ah/kg) besitzt. Bei Nickel-Cadmium Akkumulatoren besteht das Elektorlyt aus Kalilauge. Die typische Nennspannung ist exakt die selbe wie bei NiMH Akkus, 1,2V. Aus dieser Zellenspannung ergibt sich eine spezifische Energie von ungefähr 60Wh/kg. Eine Eigenschaft die man bei anderen Technologien nur selten antrifft ist das hervorragende Tieftemperaturverhalten von NiCd Akkus. Selbst bei einer Temperatur von -40°C ist eine Inbetriebnahme noch möglich. Im Jahr 2004 wurde jedoch die Verwendung von Nickel-Cadmium Akkus wegen dem giftigen Material auf medizinische und sicherheitrelevante Bereiche begrenzt. Diese Akkumulatoren sind in 2 verschiedenen Bauformen verfügbar, die sich durch die unterschiedlichen Anwendungsbereiche unterscheiden. Die offene Bauweise wird meist für Starterbatterien für Verbrennungsmotoren und Traktionsbatterien für Elektrofahrzeuge verwendet. Bei der anderen Bauweise, werden die Zellen gasdicht verschlossen. Oftmals werden sich für zentrale Stromversorgungssysteme für Notbeleuchtung verwendet.

\subsubsection{Lithium-Ionen Batterie}
