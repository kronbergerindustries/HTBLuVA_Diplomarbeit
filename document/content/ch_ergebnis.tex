\chapter{Endergebnis}
\fancyfoot[C]{Kronberger, Lackner, Kern, Schmeisser}

Zum Abgabezeitpunkt befindet sich das Motorrad noch im selben Zustand, wie nach der Ausschlachtung. Alle Konzepte für die Fertigung und Fertigstellung des Motorrades sind fertig geplant und werden zu diesem Zeitpunkt auch schon Durchgeführt.

Die Zentralsteuerung ist mittlerweile schon einsatzbereit und kann sobald der Prototyp fertig ist eingesetzt werden. Die Benutzeroberfläche ist am selben Stand und konnte schon simuliert werden. 

Ebenfalls schon fertig ist das Antriebskonzept. Motor und Motorsteuerung kommunizieren und der Motor ist von der Motorsteuerung steuerbar und Einsatzbereit, welches ebenfalls schon simuliert werden konnte. Sobald das Getriebe fertiggestellt wurde, kann der Motor auch schon montiert werden. 

Das Akku und Ladekonzept ist ebenfalls auf einem guten Weg und mit dem Bau der Akkuboxen kann nach Fertigstellung des Getriebes begonnen werden.

Der Prototypen wird aufgrund vieler Probleme mit Sponsorensuche und späten Zusagen, beziehungsweise sehr oft Absagen, um 1 bis 2 Monaten später fahrbereit sein.

Wie unter der Unterüberschrift \ref{Akkukühlung} Akkukühlung zu lesen ist, sind wie hier auch bei anderen Themen Zeit und Geld sparende Methoden angewendet worden um innerhalb eines Schuljahres fertig zu werden. Nach Schulende am 30. April wird der Prototyp fertiggestellt und in der Studienzeit noch erweitert werden, um möglicherweise die Straßenzulassung zu erhalten. Mit der Straßenzulassung sollen bestmögliche Optimierungen vorgenommen werden m auch eine Kühlung der Steuerung und damit einen sicheren Betrieb versichern zu können. Erweiterungen stehen immer offen und werden auch vorgenommen werden.

Diese Projekt ist deshalb sehr interessant, weil die Mobilbranche in diese Richtung expandieren und Techniker für diese Entwicklung benötigen wird. Neben möglicher Interessen von Firmen an uns, soll diese Projekt auch ein Forschungsobjekt für private Interessen sein, die in der Wirtschaft möglicherweise einmal von Bedeutung sein können. Diese Projekt ist ein weiterer kleiner Schritt in die Zukunft. Mit einer möglichen Verbindung des Projektes der Wasserstoffzelle eines Kollegen, kann eine Art der Fortbewegung erschaffen werde, die diesen Planeten schützen kann. 

