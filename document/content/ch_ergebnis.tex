\chapter{Endergebnis}
\fancyfoot[C]{Kronberger, Lackner, Kern, Schmeisser}

Zum Abgabezeitpunkt befindet sich das Motorrad noch im selben Zustand, wie nach der Ausschlachtung. Alle Konzepte für die Fertigung und Fertigstellung des Motorrades sind fertig geplant und werden zu diesem Zeitpunkt auch schon durchgeführt.

Die Zentralsteuerung ist mittlerweile schon einsatzbereit und kann sobald der Prototyp fertig ist eingesetzt werden. Die Benutzeroberfläche ist am selben Stand und konnte schon simuliert werden. 

Ebenfalls schon fertig ist das Antriebskonzept. Motor und Motorsteuerung kommunizieren und der Motor ist von der Motorsteuerung steuerbar und einsatzbereit, was ebenfalls schon simuliert werden konnte. Sobald das Getriebe fertiggestellt wurde, kann der Motor montiert werden. 

Das Akku und Ladekonzept ist auf einem guten Weg und mit dem Bau der Akkuboxen kann nach Fertigstellung des Getriebes begonnen werden.

Der Prototyp wird aufgrund vieler Probleme mit Sponsorensuche und späten Zusagen, beziehungsweise mehrfacher Absagen, um 1 bis 2 Monaten später fahrbereit sein.

Wie unter der Unterüberschrift \ref{Akkukühlung} Akkukühlung zu lesen ist, sind wie hier auch bei anderen Themen Zeit und Geld sparende Methoden angewendet worden um innerhalb eines Schuljahres fertig zu werden. Nach Schulende am 30. April wird der Prototyp fertiggestellt und in der Studienzeit noch erweitert werden, um möglicherweise die Straßenzulassung zu erhalten. Mit der Straßenzulassung sollen bestmögliche Optimierungen vorgenommen werden m auch eine Kühlung der Steuerung und damit einen sicheren Betrieb gewährleisten zu können. Erweiterungsmöglichkeiten stehen offen und können vorgenommen werden. 

Vor allem die detaillierte Planung im Vorhinein viel uns sehr schwer. Nicht nur weil uns oft wichtige Komponenten fehlten, um die Planung fortzusetzen. Doch oft war es einfach die Unwissenheit über das Ausmaß, welches eine zu wenig geplante Arbeit, im Nachhinein haben kann. Denn meist wären uns sehr viele Arbeitsstunden erspart geblieben, hätten wir nur genug zeit und Arbeit in die Planung der Umsetzung gesteckt.

Diese Projekt ist deshalb sehr interessant, weil die Mobilbranche in diese Richtung expandieren und Techniker für diese Entwicklung benötigen wird. Neben möglichen Interessen von Firmen an uns, soll diese Projekt auch ein Forschungsobjekt für private Interessen sein, welches für in der Wirtschaft möglicherweise einmal von Bedeutung sein könnten. Dieses Projekt ist ein weiterer kleiner Schritt in die Zukunft. Mit einer möglichen Verbindung des Projektes der Wasserstoffzelle eines Kollegen, kann eine Art der Fortbewegung erschaffen werde, die diesen Planeten schützen kann. 

