%% Eidestattliche Erklärung
\begin{center}
\begin{huge}
\textbf{Eidesstaatliche Erklärung}
\end{huge}
\end{center}

\bigskip
\justifying{
Wir erklären an Eides statt, dass wir die vorliegende Arbeit selbstständig und ohne fremde
Hilfe verfasst, andere als die angegebenen Quellen nicht benutzt und die den benutzten Quellen
wörtlich und inhaltlich entnommenen Stellen als solche kenntlich gemacht haben. Wir versichern,
dass wir dieses Diplomarbeitsthema bisher weder im In- noch im Ausland (einer Beurteilerin
oder einem Beurteiler) in irgendeiner Form als Prüfungsarbeit vorgelegt haben.}
\vspace{2cm}

%% Gender Erklärung
\begin{center}
\begin{huge}
\textbf{Gendererklärung}
\end{huge}
\end{center}

\bigskip
\justifying{
Aus Gründen der besseren Lesbarkeit wird in dieser Diplomarbeit die Sprachform
des generischen Maskulinums angewendet. Es wird an dieser Stelle darauf hingewiesen,
dass die ausschließliche Verwendung der männlichen Form geschlechtsunabhängig
verstanden werden soll.}
\vspace{2cm}

%% Unterschriften
\begin{center}
\noindent
\begin{tabular}{ll}
\makebox[2.5in]{\hrulefill} & \makebox[2.5in]{\hrulefill}\\
Martin Kronberger & Ort, Datum\\[8ex]
\makebox[2.5in]{\hrulefill} & \makebox[2.5in]{\hrulefill}\\
Jakob Lackner & Ort, Datum\\[8ex]
\makebox[2.5in]{\hrulefill} & \makebox[2.5in]{\hrulefill}\\
Simon Kern & Ort, Datum\\[8ex]
\makebox[2.5in]{\hrulefill} & \makebox[2.5in]{\hrulefill}\\
Tobias Schmeisser & Ort, Datum\\[8ex]
\end{tabular}
\end{center}

%% Leere Seite
\newpage
\thispagestyle{empty}
\mbox{}
\newpage

%% Vorwort
\begin{center}
\begin{huge}
\textbf{Vorwort}
\end{huge}
\end{center}

\bigskip
\justifying{In immer mehr Großstädten werden Fahrzeuge mit Verbrennungsmotoren verboten. Viele Motorräder und Autos können nicht mehr produziert werden, da sie die immer strenger werdenden Abgasnormen nicht mehr einhalten können und das Thema der Klimaerwärmung wird immer präsenter und immer mehr Menschen versuchen ihren „carbon footprint“ zu verkleinern.
\medskip \\ 
Doch leider gibt es für Motorradfahrer zumeist keine wirklichen Möglichkeiten, um für ihr Hobby auf eine emissionsfreie Alternativen umzusteigen. Denn zumeist ist das Preis-Leistungsverhältnis, oder auch das Produkt selbst, nicht sehr einladend. Daher ist unser Ziel die Entwicklung in diesem Bereich voranzutreiben und dadurch den Markt zu vergrößern, wodurch immer mehr und bessere Produkte angeboten werden können.}

%% Leere Seite

\newpage
\thispagestyle{empty}
\mbox{}
\newpage

\begin{center}
\begin{huge}
\textbf{Danksagung}
\end{huge}
\end{center}

\justifying{Besonders für die informelle Unterstützung im Bezug auf die technische Auslegung, Konfiguration und Programmierung des Curtis Controllers möchten wir uns auch bei den Technikern (Bereich E-Mobilität) Daniel Kramser, Sebastian Bayer und Wolfgang Strasser bedanken.}
\vspace{2cm}

%% Leere Seite

\newpage
\thispagestyle{empty}
\mbox{}
\newpage

%% Abstract Deutsch

\centering
\begin{huge}
\textbf{DIPLOMARBEIT}
\end{huge}

\begin{large}
DOKUMENTATION
\end{large}

\begin{center}
\begin{table}[H]
	\begin{tabular}{|l|l|l|}
		\hline
		Namen der Verfasser &
		\multicolumn{2}{l|}{\begin{tabular}[c]{@{}l@{}}Martin Kronberger, Jakob Lackner,\\ Simon Kern, Tobias Schmeisser\end{tabular}} \\ \hline
		\begin{tabular}[c]{@{}l@{}}Jahrgang\\ Schuljahr\end{tabular} &
		\multicolumn{2}{l|}{\begin{tabular}[c]{@{}l@{}}5AHET\\ 2020/21\end{tabular}} \\ \hline
		Thema der Diplomarbeit &
		\multicolumn{2}{l|}{Entwicklung eines emissionsfreien Sportmotorrades} \\ \hline
		Kooperationspartner &
		\multicolumn{2}{l|}{Schrack for Students, Sigmatek EBG GmbH,} \\ \hline
		Aufgabenstellung &
		\multicolumn{2}{l|}{\begin{tabular}[c]{@{}l@{}}Es soll ein vollständig elektrifiziertes E–Motorrad aus einen\\ alten Model mit Verbrennungsmotor entwickelt werden.\\ Der vorherige Verbrennungsmotor wird durch eine\\ elektrische Steuer und Motoreinheit ersetzt. Ebenfalls soll\\ das Zweirad über ein eigenständiges Akkusystem, mit einem\\ für das Motorrad individuellem Ladesystem verfügen. Über\\ eine zentrale Steuereinheit wird die Kommunikation\\ zwischen den Systemen und dem Menschen gewährleistet.\end{tabular}} \\ \hline
		Realisierung &
		\multicolumn{2}{l|}{\begin{tabular}[c]{@{}l@{}}Als Chassie wird eine ausgeschlachtete Duacati Monster S4 2001\\ verwendet und mit einem eigensentwickelten Rahmen versehen.\\ Angetrieben wird das Motorrad  über eine bürstenlose\\ Synchronmaschine, welche über ein BMS gesteuertes 50,4 Volt\\ Lithiumionen Akkupack versorgt wird. Das Moment wird vom\\ Motor über ein Kettengetriebe mit eine Übersetzung von\\ ungefähr 1/9 auf die Straße übertragen. Die Peripherie wird über\\ einen Raspberry Pi Minicomputer und Taster am Lenker gesteuert.\\ Ebenso wird über ihn die Benutzeroberfläche gesteuert, welche\\ über ein 11,6 Zoll Touch Panel angezeigt und gesteuert werden kann.\end{tabular}} \\ \hline
		Ergebnisse &
		\multicolumn{2}{l|}{\begin{tabular}[c]{@{}l@{}}Zum Abgabezeitpunkt befindet sich die Komponenten des Motorrads\\ im Entwicklungszustand. Der Rahmen ist noch in der Endphase der\\ Fertigung. Der Motor und dessen Software ist Einsatzbereit und\\ müsste nur mehr eingebaut werden. Die Versorgung ist zu ende\\ entwickelt und muss ebenso nur mehr gefertigt werden. Die\\ Steuerung der Benutzeroberfläche befindet sich im Prototypen-Status,\\ beinhaltet jedoch noch einige experimentelle Funktionen, welche noch\\ etwas Programmier und Testzeit benötigen. Gesamt ist das gesamte\\ Konzept vollständig und kann mit etwas mehr Zeit und Sponsorengelder\\ fertiggestellt werden.\end{tabular}} \\ \hline
		\begin{tabular}[c]{@{}l@{}}Möglichkeit der Einsicht-\\ nahme in die Arbeit\end{tabular} &
		\multicolumn{2}{l|}{\begin{tabular}[c]{@{}l@{}}Die Diplomarbeit ist in gebundener Form sowohl in der Schulbibliothek\\ als auch bei AV Prof. Dipl-Ing. (FH) Roland Holzer einzusehen.\\ Darüber hinaus besitzt jedes Mitglied des Projektteams eine vollständige\\ Version in gebundener und digitaler Form.\end{tabular}} \\ \hline
		\begin{tabular}[c]{@{}l@{}}Approbation\\ (Datum/Unterschrift)\end{tabular} &
		Prüfer &
		Abteilungsvorstand \\ \hline
	\end{tabular}
\end{table}
\end{center}
%% Abstract Englisch

\newpage
\centering
\begin{huge}
\textbf{DIPLOMA THESIS}
\end{huge}

\begin{large}
DOCUMENTATION
\end{large}

\begin{center}
\begin{table}[H]
	\begin{tabular}{|l|l|l|}
		\hline
		Author(s) &
		\multicolumn{2}{l|}{\begin{tabular}[c]{@{}l@{}}Martin Kronberger, Jakob Lackner,\\ Simon Kern, Tobias Schmeisser\end{tabular}} \\ \hline
		\begin{tabular}[c]{@{}l@{}}Form\\ Academic year\end{tabular} &
		\multicolumn{2}{l|}{\begin{tabular}[c]{@{}l@{}}5AHET\\ 2020/21\end{tabular}} \\ \hline
		Topic &
		\multicolumn{2}{l|}{Development of an emission-free sports motorcycle} \\ \hline
		Co-operation partners &
		\multicolumn{2}{l|}{Schrack for Students, Sigmatek EBG GmbH,} \\ \hline
		Assignment of Tasks &
		\multicolumn{2}{l|}{\begin{tabular}[c]{@{}l@{}}The plan is to develop a fully electrified e-motorcycle.\\ The previous internal combustion engine is replaced by\\ an control unit and electric motor unit. The two-wheeler\\ should also have an independent battery system with a\\ charging system that is individual for the motorcycle.\\ Communication between the systems and humans is\\ ensured via a central control unit.\end{tabular}} \\ \hline
		Realisation &
		\multicolumn{2}{l|}{\begin{tabular}[c]{@{}l@{}}A cannibalized Duacati Monster S4 2001 is used as\\ the chassis and provided with a specially developed\\ frame. The motorcycle is driven by a brushless\\ synchronous machine, which is supplied by a BMS\\ controlled 50.4 volt lithium ion battery pack. The\\ torque is transmitted from the engine to the road\\ via a chain gear with a ratio of  about 1/9. The\\ periphery is controlled via a Raspberry Pi\\ minicomputer and button on the handlebar.\\ It also controls the user interface, which can be\\ displayed and controlled via an 11.6 inch touch panel\end{tabular}} \\ \hline
		Results &
		\multicolumn{2}{l|}{\begin{tabular}[c]{@{}l@{}}The components of the motorcycle are at the time of\\ delivery in the state of development. The framework \\ is still in its final stages Production. The engine and\\ its software are ready for use and would just have to\\ be built in. The supply is over developed and just\\ needs to be manufactured. The Control of the user\\ interface is in the prototype status, however still\\ contains some experimental functions which are\\ still unavailable\end{tabular}} \\ \hline
		\begin{tabular}[c]{@{}l@{}}Accessibility of Diploma\\ Thesis\end{tabular} &
		\multicolumn{2}{l|}{\begin{tabular}[c]{@{}l@{}}The diploma thesis is available in the school library and at\\ Prof. Dipl.-Ing (FH) Roland Holzer's office. Furthermore,\\ each member of the project team has a complete version.\end{tabular}} \\ \hline
		\begin{tabular}[c]{@{}l@{}}Approval\\ (Date/Sign)\end{tabular} &
		Examiner &
		Department Manager \\ \hline
	\end{tabular}
\end{table}
\end{center}

%% Leere Seite

\newpage
\thispagestyle{empty}
\mbox{}
\newpage

%% Erklärung

\newpage
\begin{huge}
\textbf{Erklärung}
\end{huge}
\bigskip

\justifying{Die unterfertigten Kandidaten haben gemäß \S 34 (3) SchUG in Verbindung mit \S 22 (1) Zi. 3 lit. b der Verordnung über die abschließenden Prüfungen in den berufsbildenden mittleren und höheren Schulen, BGBl. II Nr. 70 vom 24.02.2000 (Prüfungsordnung BMHS), die Ausarbeitung einer Diplomarbeit mit der umseitig angeführten Aufgabenstellung gewählt. Die Kandidaten nehmen zur Kenntnis, dass die Diplomarbeit in eigenständiger Weise und außerhalb des Unterrichtes zu bearbeiten und anzufertigen ist, wobei Ergebnisse des Unterrichtes mit einbezogen werden können. Die Abgabe der vollständigen Diplomarbeit hat bis spätestens}

\bigskip
\begin{center}
\begin{large}
03.04.2020
\end{large}
\end{center}
\bigskip

\justifying{beim zuständigen Betreuer zu erfolgen. Die Kandidaten nehmen weiters zur Kenntnis, dass gemäß \S 9 (6) der Prüfungsordnung BMHS nur der Schulleiter bis spätestens Ende des vorletzten Semesters den Abbruch einer Diplomarbeit anordnen kann, wenn diese aus nicht beim Prüfungskandidaten / bei den Prüfungskandidaten gelegenen Gründen nicht fertiggestellt werden kann.}

%% Unterschriften Schüler
\newpage
\begin{small}
\begin{center}
\begin{tabular}{|p{6cm}|p{8cm}|}
\hline
\renewcommand{\arraystretch}{2}
& \\
\textbf{Kandidaten / Kandidatinnen} & \textbf{Unterschrift} \\ 
& \\ \hline
& \\
Martin Kronberger &  \\ 
& \\ \hline
& \\
Jakob Lackner &  \\ 
& \\ \hline
& \\
Simon Kern &  \\ 
& \\ \hline
& \\
Tobias Schmeisser &  \\ 
& \\ \hline
\end{tabular}
\end{center}
\end{small}
\vspace{3cm}

%% Unterschriften Lehrer
\begin{small}
\begin{center}
\noindent
\begin{tabular}{cc}
\makebox[6.35cm]{\hrulefill} & \makebox[6.35cm]{\hrulefill}\\
Prof. Dipl.-Ing. Reinhold Benedikter & Prof. Dipl.-Ing. (FH) Johannes Ferner\\
Prüfer & Prüfer\\[18ex]
\makebox[6.35cm]{\hrulefill} & \makebox[6.35cm]{\hrulefill}\\
Prof. Dipl.-Ing. Adolf Reinhart, MBA  & Prof. Dipl.-Ing. Peter Lindmoser\\
Prüfer & Prüfer\\[18ex]
\makebox[6.35cm]{\hrulefill} & \makebox[6.35cm]{\hrulefill}\\
Prof. Dipl.-Ing. (FH) Roland Holzer & Dipl.-Ing. Dr.tech. Franz Landertshamer\\
Abteilungsvorstand & Direktor\\[18ex]
\end{tabular}
\end{center}
\end{small}
\raggedright