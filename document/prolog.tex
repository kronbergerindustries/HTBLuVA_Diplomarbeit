%% Eidestattliche Erklärung
\begin{center}
\begin{huge}
\textbf{Eidesstaatliche Erklärung}
\end{huge}
\end{center}

\bigskip
\justifying{
Wir erklären an Eides statt, dass wir die vorliegende Arbeit selbstständig und ohne fremde
Hilfe verfasst, andere als die angegebenen Quellen nicht benutzt und die den benutzten Quellen
wörtlich und inhaltlich entnommenen Stellen als solche kenntlich gemacht haben. Wir versichern,
dass wir dieses Diplomarbeitsthema bisher weder im In- noch im Ausland (einer Beurteilerin
oder einem Beurteiler) in irgendeiner Form als Prüfungsarbeit vorgelegt haben.}
\vspace{2cm}

%% Gender Erklärung
\begin{center}
\begin{huge}
\textbf{Gendererklärung}
\end{huge}
\end{center}

\bigskip
\justifying{
Aus Gründen der besseren Lesbarkeit wird in dieser Diplomarbeit die Sprachform
des generischen Maskulinums angewendet. Es wird an dieser Stelle darauf hingewiesen,
dass die ausschließliche Verwendung der männlichen Form geschlechtsunabhängig
verstanden werden soll.}
\vspace{2cm}

%% Unterschriften
\begin{center}
\noindent
\begin{tabular}{ll}
\makebox[2.5in]{\hrulefill} & \makebox[2.5in]{\hrulefill}\\
Martin Kronberger & Ort, Datum\\[8ex]
\makebox[2.5in]{\hrulefill} & \makebox[2.5in]{\hrulefill}\\
Jakob Lackner & Ort, Datum\\[8ex]
\makebox[2.5in]{\hrulefill} & \makebox[2.5in]{\hrulefill}\\
Simon Kern & Ort, Datum\\[8ex]
\makebox[2.5in]{\hrulefill} & \makebox[2.5in]{\hrulefill}\\
Tobias Schmeisser & Ort, Datum\\[8ex]
\end{tabular}
\end{center}

%% Leere Seite
\newpage
\thispagestyle{empty}
\mbox{}
\newpage

%% Vorwort
\begin{center}
\begin{huge}
\textbf{Vorwort}
\end{huge}
\end{center}

\bigskip
\justifying{VORWORT}

%% Leere Seite

\newpage
\thispagestyle{empty}
\mbox{}
\newpage

\begin{center}
\begin{huge}
\textbf{Danksagung}
\end{huge}
\end{center}

\justifying{TEXT DANKSAGUNG}
\vspace{2cm}

%% Leere Seite

\newpage
\thispagestyle{empty}
\mbox{}
\newpage

%% Abstract Deutsch

\centering
\begin{huge}
\textbf{DIPLOMARBEIT}
\end{huge}

\begin{large}
DOKUMENTATION
\end{large}

\begin{tabular}{|l|l|l|}
\hline
 &  &  \\ \hline
 &  &  \\ \hline
 &  &  \\ \hline
 &  &  \\ \hline
 &  &  \\ \hline
 &  &  \\ \hline
 &  &  \\ \hline
 &  &  \\ \hline
 &  &  \\ \hline
\end{tabular}

%% Abstract Englisch

\newpage
\centering
\begin{huge}
\textbf{DIPLOMA THESIS}
\end{huge}

\begin{large}
DOCUMENTATION
\end{large}

\begin{tabular}{|l|l|l|}
\hline
 &  &  \\ \hline
 &  &  \\ \hline
 &  &  \\ \hline
 &  &  \\ \hline
 &  &  \\ \hline
 &  &  \\ \hline
 &  &  \\ \hline
 &  &  \\ \hline
 &  &  \\ \hline
\end{tabular}

%% Einreichungsunterlagen

\newpage
\begin{huge}
\textbf{Einreichungsunterlagen}
\end{huge}

\begin{large}
5AHET Reife und Diplomarbeitsprüfung 2020/21
\end{large}

\begin{tabular}{|l|l|l|}
\hline
 &  &  \\ \hline
 &  &  \\ \hline
 &  &  \\ \hline
 &  &  \\ \hline
 &  &  \\ \hline
 &  &  \\ \hline
 &  &  \\ \hline
 &  &  \\ \hline
 &  &  \\ \hline
\end{tabular}

%% Leere Seite

\newpage
\thispagestyle{empty}
\mbox{}
\newpage

%% Erklärung

\newpage
\begin{huge}
\textbf{Erklärung}
\end{huge}
\bigskip

\justifying{Die unterfertigten Kandidaten haben gemäß \S 34 (3) SchUG in Verbindung mit \S 22 (1) Zi. 3 lit. b der Verordnung über die abschließenden Prüfungen in den berufsbildenden mittleren und höheren Schulen, BGBl. II Nr. 70 vom 24.02.2000 (Prüfungsordnung BMHS), die Ausarbeitung einer Diplomarbeit mit der umseitig angeführten Aufgabenstellung gewählt. Die Kandidaten nehmen zur Kenntnis, dass die Diplomarbeit in eigenständiger Weise und außerhalb des Unterrichtes zu bearbeiten und anzufertigen ist, wobei Ergebnisse des Unterrichtes mit einbezogen werden können. Die Abgabe der vollständigen Diplomarbeit hat bis spätestens}

\bigskip
\begin{center}
\begin{large}
03.04.2020
\end{large}
\end{center}
\bigskip

\justifying{beim zuständigen Betreuer zu erfolgen. Die Kandidaten nehmen weiters zur Kenntnis, dass gemäß \S 9 (6) der Prüfungsordnung BMHS nur der Schulleiter bis spätestens Ende des vorletzten Semesters den Abbruch einer Diplomarbeit anordnen kann, wenn diese aus nicht beim Prüfungskandidaten / bei den Prüfungskandidaten gelegenen Gründen nicht fertiggestellt werden kann.}

%% Unterschriften Schüler
\newpage
\begin{small}
\begin{center}
\begin{tabular}{|p{6cm}|p{8cm}|}
\hline
\renewcommand{\arraystretch}{2}
& \\
\textbf{Kandidaten / Kandidatinnen} & \textbf{Unterschrift} \\ 
& \\ \hline
& \\
Martin Kronberger &  \\ 
& \\ \hline
& \\
Jakob Lackner &  \\ 
& \\ \hline
& \\
Simon Kern &  \\ 
& \\ \hline
& \\
Tobias Schmeisser &  \\ 
& \\ \hline
\end{tabular}
\end{center}
\end{small}
\vspace{3cm}

%% Unterschriften Lehrer
\begin{small}
\begin{center}
\noindent
\begin{tabular}{cc}
\makebox[6.35cm]{\hrulefill} & \makebox[6.35cm]{\hrulefill}\\
Prof. Dipl.-Ing. Reinhold Benedikter & Dipl.-Ing. (FH) Johannes Ferner\\
Prüfer & Prüfer\\[18ex]
\makebox[6.35cm]{\hrulefill} & \makebox[6.35cm]{\hrulefill}\\
Prof. Dipl.-Ing. MBA Adolf Reinhart & Lindmoser, Prof. Dipl.-Ing. Peter\\
Prüfer & Prüfer\\[18ex]
\makebox[6.35cm]{\hrulefill} & \makebox[6.35cm]{\hrulefill}\\
Prof. Dipl.-Ing. (FH) Roland Holzer & Dipl.-Ing. Dr.techn. Franz Landertshamer\\
Abteilungsvorstand & Direktor\\[18ex]
\end{tabular}
\end{center}
\end{small}
\raggedright