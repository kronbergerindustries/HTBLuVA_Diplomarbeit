%% Eidestattliche Erklärung
\begin{center}
\begin{huge}
\textbf{Eidesstaatliche Erklärung}
\end{huge}
\end{center}

\bigskip
\justifying{
Wir erklären an Eides statt, dass wir die vorliegende Arbeit selbstständig und ohne fremde
Hilfe verfasst, andere als die angegebenen Quellen nicht benutzt und die den benutzten Quellen
wörtlich und inhaltlich entnommenen Stellen als solche kenntlich gemacht haben. Wir versichern,
dass wir dieses Diplomarbeitsthema bisher weder im In- noch im Ausland (einer Beurteilerin
oder einem Beurteiler) in irgendeiner Form als Prüfungsarbeit vorgelegt haben.}
\vspace{2cm}

%% Gender Erklärung
\begin{center}
\begin{huge}
\textbf{Gendererklärung}
\end{huge}
\end{center}

\bigskip
\justifying{
Aus Gründen der besseren Lesbarkeit wird in dieser Diplomarbeit die Sprachform
des generischen Maskulinums angewendet. Es wird an dieser Stelle darauf hingewiesen,
dass die ausschließliche Verwendung der männlichen Form geschlechtsunabhängig
verstanden werden soll.}
\vspace{2cm}

%% Unterschriften
\begin{center}
\noindent
\begin{tabular}{ll}
\makebox[2.5in]{\hrulefill} & \makebox[2.5in]{\hrulefill}\\
Martin Kronberger & Ort, Datum\\[8ex]
\makebox[2.5in]{\hrulefill} & \makebox[2.5in]{\hrulefill}\\
Jakob Lackner & Ort, Datum\\[8ex]
\makebox[2.5in]{\hrulefill} & \makebox[2.5in]{\hrulefill}\\
Simon Kern & Ort, Datum\\[8ex]
\makebox[2.5in]{\hrulefill} & \makebox[2.5in]{\hrulefill}\\
Tobias Schmeisser & Ort, Datum\\[8ex]
\end{tabular}
\end{center}

%% Leere Seite
\newpage
\thispagestyle{empty}
\mbox{}
\newpage

%% Vorwort
\begin{center}
\begin{huge}
\textbf{Vorwort}
\end{huge}
\end{center}

\bigskip
\justifying{In immer mehr Großstädten werden Fahrzeuge mit Verbrennungsmotoren verboten. Viele Motorräder und Autos können nicht mehr produziert werden, da sie die immer strenger werdenden Abgasnormen nicht mehr einhalten können und das Thema der Klimaerwärmung wird immer präsenter und immer mehr Menschen versuchen ihren „carbon footprint“ zu verkleinern.
\medskip \\ 
Doch leider gibt es für Motorradfahrer zumeist keine wirklichen Möglichkeiten, um für ihr Hobby auf eine emissionsfreie Alternativen umzusteigen. Denn zumeist ist das Preis-Leistungsverhältnis, oder auch das Produkt selbst, nicht sehr einladend. Daher ist unser Ziel die Entwicklung in diesem Bereich voranzutreiben und dadurch den Markt zu vergrößern, wodurch immer mehr und bessere Produkte angeboten werden können.}

%% Leere Seite

\newpage
\thispagestyle{empty}
\mbox{}
\newpage

\begin{center}
\begin{huge}
\textbf{Danksagung}
\end{huge}
\end{center}

Für mechanische Komponenten, möchten wir uns bei den Firmen Theodor Wimmer GmbH aus Hof bei Salzburg und G.O. Nilsson Ges.m.b.H. SKF - Vetragshändler aus Wals bedanken. Bei der Firma Theodor Wimmer GmbH möchten wir uns ganz besonders bei Gerlinde und Theodor Wimmer bedanken, welche uns nicht nur mit Bauteilen unterstützen, sondern uns auch bei der Suche nach weiteren Firmen für dieses kostspielige Unternehemen, tatkräftig unterstützten und für Fragen jederzeit erreichbar waren. Bei der Firma G.O. Nilsson Ges.m.b.H. möchten wir Herrn Michael Kudrna nicht unerwähnt lassen, welcher die Ansprechperson für Tobias Schmeisser bei diversen Bauteilen für das Getriebe war.

\begin{figure} [H]
	\begin{center}
		\includegraphics[width=10cm]{figures/mechanik/Sponsoren.jpg}		
	\end{center}
\end{figure}
	
Besonders für die informelle Unterstützung im Bezug auf die technische Auslegung, Konfiguration und Programmierung des Curtis Controllers möchten wir uns auch bei den Technikern (Bereich E-Mobilität) Daniel Kramser, Sebastian Bayer und Wolfgang Strasser bedanken.

Für die finanzielle Unterstützung und Beschaffung von Bauteilen bedanken wir uns auch bei der Firma Schrack for Students.

\begin{figure}[H]
	\begin{center}
		\includegraphics[width=10cm]{figures/allgemein/Schrack_Logo.jpg}
		\caption{Sigmatek Logo}
	\end{center}
\end{figure}

\begin{figure}[H]
	\begin{center}
		\includegraphics[width=10cm]{figures/allgemein/Sigmatek_LOGO.jpg}
		\caption{Sigmatek Logo}
	\end{center}
\end{figure}

\vspace{1cm}

Besonders bei der Firma ELIN GmbH möchten wir uns für die finazielle Unterstützung bedanken. Spezielle möchte ich mich bei Ing. Thomas Meister, für die informelle Unterstützung in Bezug auf das gesamte Akkusystem, bedanken.
Ebenfalls möchte ich mich bei meinem Projektbetreuer Prof. Dipl.-Ing. Reinhold Benedikter für die Untertützung, im gesamten Verlauf meiner Diplomarbeit, bedanken.

\begin{figure}[H]
	\begin{center}
		\includegraphics[width=10cm]{figures/allgemein/Elin Logo.png}
	\end{center}
\end{figure}
\vspace{1cm}

Zu Beginn möchten wir uns bei unseren Betreuen besonders bedanken. Prof. Dipl.-Ing. MBA Adolf Reinhart unterstütze Jakob Lackner in vielen Bereichen, unter anderem konnte er ihm in Bezug auf den Elektromotor fachlich immer weiterhelfen. Prof. Dipl.-Ing. Reinhold Benedikter könnte Simon Kern bei der Dimensionierung und Recherche, im Bereich des Akkusystems, besonders weiterhelfen. Außerdem war Herr Benedikter bereit allerlei Fragen zu beantworten. und bereit uns weiterzuhelfen. Prof. Dipl.-Ing. Peter Lindmoser unterstützte Tobias Schmeisser bei der mechanischen Umsetzung des Konzepts und war für jegliche Fragen immer offen

Zu Beginn mochten wir uns bei unseren Betreuern bedanken. Prof. Dipl.-Ing. Jakob Mühlbacher
unterstutzte uns in vielen Bereichen, unter anderem konnte er uns fachlich immer weiterhelfen
und zeigte ein großes Interesse an unserer Diplomarbeit. Sein enger Kontakt zur Salzburg AG
durch Ing. Johann Schmidhuber war von großer Bedeutung.
Prof. Dipl.-Ing. Reinhold Benedikter zeigte großes Interesse an unserem Projekt und half uns
immer wieder durch innovative Vorschläge und seinem Fachwissen.
Einen großen Anteil der Verwirklichung unseres Projektes ist Ing. Johann Schmidhuber anzurechnen. Neben der Finanziellen Unterstützung durch die Salzburg AG hatte er auch immer wieder
ein offenes Ohr für uns bei etwaigen Problemen

Dipl.-Ing. (FH) Johannes Ferner
Prof. Dipl.-Ing. MBA Adolf Reinhart
Prof. Dipl.-Ing. Reinhold Benedikter
Prof. Dipl.-Ing. Peter Lindmoser

%% Leere Seite

\newpage
\thispagestyle{empty}
\mbox{}
\newpage

%% Abstract Deutsch



{
	
	\newcommand{\tabitem}{~~\llap{\textbullet}~~}
	\newenvironment{mytable}[1][{|X[1,c,m]|X[2.1,l,m]|}]{
		\begin{tabu} to \textwidth {#1}
			\hline
		}{
			
		\end{tabu}
	}

	\centering
	\begin{huge}
		\textbf{DIPLOMARBEIT}
	\end{huge}
	
	\begin{large}
		DOKUMENTATION
	\end{large}
	
	\tabulinesep = 2.5mm
	\let\cleardoublepage\clearpage
	%DATE 2.4 12:01
	\begin{hyphenrules}{nohyphenation}
		\begin{center}
			\begin{mytable}
				Namen der Verfasser/innen &
				Martin Kronberger (Projektleiter), Jakob Lackner, Simon Kern, Tobias Schmeisser \\
				\hline
				Jahrgang / Schuljahr &
				$5$AHET, $2020/2021$\\
				\hline
				Thema der Diplomarbeit & Entwicklung eines emissionsfreien Sportmotorrades \\
				\hline
				Kooperationspartner &
				Schrack for Students, Sigmatek GmbH \& Co KG, ALIN GmbH / EBG GmbH, G. O. Nilsson Ges.m.b.H., Theodor Wimmer GmbH  \\
				\hline
				Aufgabenstellung & Es soll ein vollständig elektrifiziertes E–Motorrad aus einen alten Model mit Verbrennungsmotor entwickelt werden. Der vorherige Verbrennungsmotor wird durch eine elektrische Steuer und Motoreinheit ersetzt. Ebenfalls soll das Zweirad über ein eigenständiges Akkusystem, mit einem für das Motorrad individuellem Ladesystem verfügen. Über eine zentrale Steuereinheit wird die Kommunikation zwischen den Systemen und dem Menschen gewährleistet.\\
				\hline
				Realisierung & ls Chassie wird eine ausgeschlachtete Duacati Monster S4 2001 verwendet und mit einem eigensentwickelten Rahmen versehen. Angetrieben wird das Motorrad  über eine bürstenlose Synchronmaschine, welche über ein BMS gesteuertes 50,4 Volt Lithiumionen Akkupack versorgt wird. Das Moment wird vom Motor über ein Kettengetriebe mit eine Übersetzung von ungefähr 1/9 auf die Straße übertragen. Die Peripherie wird übereinen Raspberry Pi Minicomputer und Taster am Lenker gesteuert. Ebenso wird über ihn die Benutzeroberfläche gesteuert, welche über ein 11,6 Zoll Touch Panel angezeigt und gesteuert werden kann.\\
				\hline
			\end{mytable}
			
			\begin{mytable}
				Ergebnisse & Zum Abgabezeitpunkt befindet sich die Komponenten des Motorrads im Entwicklungszustand. Der Rahmen ist noch in der Endphase der Fertigung. Der Motor und dessen Software sind einsatzbereit und müsste nur mehr eingebaut werden.Die Versorgung ist fertig entwickelt und muss ebenso nur mehr gefertigt werden. Die Steuerung der Benutzeroberfläche befindet sich im Prototypen-Status, beinhaltet jedoch noch einige experimentelle Funktionen, welche noch etwas Programmier- und Testzeit benötigen. Das gesamte Konzept ist vollständig und kann mit etwas mehr Zeit und Sponsorengelder fertiggestellt werden. \\
			\end{mytable}\vskip-0.42cm
			\begin{mytable}
				Möglichkeit der Einsichtnahme in die Arbeit & Die Diplomarbeit ist in gebundener Form sowohl in der Schulbibliothek als auch bei AV Prof. Dipl-Ing. (FH) Roland Holzer einzusehen. Darüber hinaus besitzt jedes Mitglied des Projektteams eine vollständige Version in gebundener und digitaler Form.\\
			\end{mytable}\vskip-0.42cm
			\begin{mytable}[{|X[0.995,c]|X[1,m]|X[1,m]|}]
				Approbation \newline (Datum / Unterschrift) &
				\hbox{\footnotesize{Prüfer:}} &
				\hbox{\footnotesize{Abteilungsvorstand:}} \\
				\hline
			\end{mytable}
		\end{center}
	\end{hyphenrules}

\newpage

	\centering
\begin{huge}
	\textbf{Diploma Thesis}
\end{huge}

\begin{large}
	 Documentation
\end{large}

	%DATE 2.4 12:01
	\begin{hyphenrules}{nohyphenation}
		\begin{center}
			\begin{mytable}
				Author(s) &
				Martin Kronberger (Project Leader), Jakob Lackner, Simon Kern, Tobias Schmeisser \\
				\hline
				Form / Academic year &
				$5$AHET, $2020/2021$\\
				\hline
				Topic &  Development of an emission-free sports motorcycle \\
				\hline
				Co-operation partners & Schrack for Students, Sigmatek GmbH \& Co KG, ALIN GmbH / EBG GmbH, G. O. Nilsson Ges.m.b.H., Theodor Wimmer GmbH \\
				\hline
				Assignment of Tasks & The plan is to develop a fully electrified e-motorcycle. The previous internal combustion engine is replaced by an control unit and electric motor unit. The two-wheeler should also have an independent battery system with a charging system that is individual for the motorcycle. Communication between the systems and humans is ensured via a central control unit.\\
				\hline
				Realisation & A cannibalized Duacati Monster S4 2001 is used as the chassis and provided with a specially developed frame. The motorcycle is driven by a brushless synchronous machine, which is supplied by a BMS controlled 50.4 volt lithium ion battery pack. The torque is transmitted from the engine to the road via a chain gear with a ratio of about 1/9. The periphery is controlled via a Raspberry Pi minicomputer and button on the handlebar. It also controls the user interface, which can be displayed and controlled via an 11.6 inch touch panel.\\
				\hline
			\end{mytable}
			
			\begin{mytable}
				Results & The components of the motorcycle are at the time of delivery in the state of development. The framework is still in its final stages Production. The engine and its software are ready for use and would just have to be built in. The supply is over developed and just needs to be manufactured. The Control of the user interface is in the prototype status, however still contains some experimental functions which are still unavailable\\
			\end{mytable}\vskip-0.42cm
			\begin{mytable}
				Accessibility of Diploma Thesis & The diploma thesis is available in the school library and at Prof. Dipl.-Ing (FH) Roland Holzer’s office. Furthermore, each member of the project team has a complete version.
				\\
			\end{mytable}\vskip-0.42cm
			\begin{mytable}[{|X[0.995,c]|X[1,m]|X[1,m]|}]
				\hfil Approval \newline \hfil (Date / Signature) &
				\hbox{\footnotesize{Examiner:}} &
				\hbox{\footnotesize{Head of Department:}} \\
				\hline
			\end{mytable}
		\end{center}
	\end{hyphenrules}
}


%% Leere Seite

\newpage
\thispagestyle{empty}
\mbox{}
\newpage

%% Erklärung

\newpage
\begin{huge}
\textbf{Erklärung}
\end{huge}
\bigskip

\justifying{Die unterfertigten Kandidaten haben gemäß \S 34 (3) SchUG in Verbindung mit \S 22 (1) Zi. 3 lit. b der Verordnung über die abschließenden Prüfungen in den berufsbildenden mittleren und höheren Schulen, BGBl. II Nr. 70 vom 24.02.2000 (Prüfungsordnung BMHS), die Ausarbeitung einer Diplomarbeit mit der umseitig angeführten Aufgabenstellung gewählt. Die Kandidaten nehmen zur Kenntnis, dass die Diplomarbeit in eigenständiger Weise und außerhalb des Unterrichtes zu bearbeiten und anzufertigen ist, wobei Ergebnisse des Unterrichtes mit einbezogen werden können. Die Abgabe der vollständigen Diplomarbeit hat bis spätestens}

\bigskip
\begin{center}
\begin{large}
03.04.2020
\end{large}
\end{center}
\bigskip

\justifying{beim zuständigen Betreuer zu erfolgen. Die Kandidaten nehmen weiters zur Kenntnis, dass gemäß \S 9 (6) der Prüfungsordnung BMHS nur der Schulleiter bis spätestens Ende des vorletzten Semesters den Abbruch einer Diplomarbeit anordnen kann, wenn diese aus nicht beim Prüfungskandidaten / bei den Prüfungskandidaten gelegenen Gründen nicht fertiggestellt werden kann.}

%% Unterschriften Schüler
\newpage
\begin{small}
\begin{center}
\begin{tabular}{|p{6cm}|p{8cm}|}
\hline
\renewcommand{\arraystretch}{2}
& \\
\textbf{Kandidaten / Kandidatinnen} & \textbf{Unterschrift} \\ 
& \\ \hline
& \\
Martin Kronberger &  \\ 
& \\ \hline
& \\
Jakob Lackner &  \\ 
& \\ \hline
& \\
Simon Kern &  \\ 
& \\ \hline
& \\
Tobias Schmeisser &  \\ 
& \\ \hline
\end{tabular}
\end{center}
\end{small}
\vspace{3cm}

%% Unterschriften Lehrer
\begin{small}
\begin{center}
\noindent
\begin{tabular}{cc}
\makebox[6.35cm]{\hrulefill} & \makebox[6.35cm]{\hrulefill}\\
Prof. Dipl.-Ing. Reinhold Benedikter & Prof. Dipl.-Ing. (FH) Johannes Ferner\\
Prüfer & Prüfer\\[18ex]
\makebox[6.35cm]{\hrulefill} & \makebox[6.35cm]{\hrulefill}\\
Prof. Dipl.-Ing. Adolf Reinhart, MBA  & Prof. Dipl.-Ing. Peter Lindmoser\\
Prüfer & Prüfer\\[18ex]
\makebox[6.35cm]{\hrulefill} & \makebox[6.35cm]{\hrulefill}\\
Prof. Dipl.-Ing. (FH) Roland Holzer & Dipl.-Ing. Dr.tech. Franz Landertshamer\\
Abteilungsvorstand & Direktor\\[18ex]
\end{tabular}
\end{center}
\end{small}
\raggedright