%% Preamble
%%%%%%%%%%%%%%%%%%%%%%%%%%%%%%%%%%%%%%%%%%%%%%%%%%%%%%%%%%%%%%

%%%%%%%%%%%%%%%%%%%%%%%%%%%%%%%%%%%%%%%%%%%%%%%%%%%%%%%%%%%%%
%% HEADER
%%%%%%%%%%%%%%%%%%%%%%%%%%%%%%%%%%%%%%%%%%%%%%%%%%%%%%%%%%%%%
\documentclass[a4paper,10pt]{report}
\usepackage{a4}

%% Deutsche Anpassungen %%%%%%%%%%%%%%%%%%%%%%%%%%%%%%%%%%%%%
\usepackage[ngerman]{babel}
\usepackage[T1]{fontenc}
\usepackage[utf8]{inputenc}

\usepackage{lmodern} %Type1-Schriftart für nicht-englische Texte

\usepackage{parskip}

%% Packages für Grafiken & Abbildungen %%%%%%%%%%%%%%%%%%%%%%
\usepackage{graphicx} %%Zum Laden von Grafiken
%\usepackage{subfig} %%Teilabbildungen in einer Abbildung
%\usepackage{tikz} %%Vektorgrafiken aus LaTeX heraus erstellen

  %% Packages für Formeln %%%%%%%%%%%%%%%%%%%%%%%%%%%%%%%%%%%%%
\usepackage{amsmath}
\usepackage{amssymb,amscd}
\usepackage{amsthm}
\usepackage{amsfonts}
\usepackage{hyperref}
\usepackage[all]{hypcap}
\usepackage{mathrsfs}
\usepackage{tabls}
\usepackage{amsthm}

\usepackage{fancyhdr} %%Fancy Kopf- und Fußzeilen
%\usepackage{longtable} %%Für Tabellen, die eine Seite überschreiten
\usepackage[text={150mm,240mm}]{geometry}

\usepackage{listings}
\usepackage{color}
%Brüche kürzen
\usepackage{cancel}

\definecolor{dkgreen}{rgb}{0,0.6,0}
\definecolor{gray}{rgb}{0.5,0.5,0.5}
\definecolor{mauve}{rgb}{0.58,0,0.82}
\definecolor{byzantium}{rgb}{0.44, 0.16, 0.39}

\lstset{frame=tb,
  language=C,
  aboveskip=3mm,
  belowskip=3mm,
  showstringspaces=false,
  columns=flexible,
  basicstyle={\small\ttfamily},
  numbers=none,
  numberstyle=\tiny\color{gray},
  keywordstyle=\color{byzantium},
  commentstyle=\color{dkgreen},
  stringstyle=\color{mauve},
  breaklines=true,
  breakatwhitespace=true,
  tabsize=3
}


%%%%%%%%%%%%%%%%%%%%%%%%%%%%%%%%%%%%%%%%%%%%%%%%%%%%%%%%%%%%%
%% DOKUMENT
%%%%%%%%%%%%%%%%%%%%%%%%%%%%%%%%%%%%%%%%%%%%%%%%%%%%%%%%%%%%%
\begin{document}

\pagestyle{empty} %%Keine Kopf-/Fusszeilen auf den ersten Seiten.

\parindent 0pt                     % Setzt die Einrückung der ersten Zeile auf 0 Pt

%% Deckblatt %%%%%%%%%%%%%%%%%%%%%%%%%%%%%%%%%%%%%%%%%%%%%%%%
% Beginn - Formatierung der Titelseite
\vspace*{20mm}
\begin{center} 
\begin{Huge}\textbf{Diplomarbeit}\end{Huge} \par \bigskip 
\begin{Large}Höhere Technische Bundes- Lehr- und Versuchsanstalt Salzburg \par \medskip 
Abteilung für Elektrotechnik\end{Large} 
\vspace*{20mm}

\begin{Huge}
\textbf{Entwicklung eines emissionsfreien Sportmotorrades \\[1ex]}
\end{Huge}
\vspace{\fill}

\begin{tabular}{lll}
\multicolumn{3}{l}{\textbf{Entwicklung der Zentralsteuerung / Projektleitung}} \\
 Martin Kronberger & 5AHET & Betreuer: Dipl.-Ing. Johannes Ferner\\
      &       &      \\
\multicolumn{3}{l}{\textbf{Entwicklung des Antriebssystems}}\\
Jakob Lackner & 5AHET & Betreuer: Prof. Dipl.-Ing. Adolf Reinhart \\
      &       &      \\
\multicolumn{3}{l}{\textbf{Entwicklung des Akkusystems}} \\
Simon Kern & 5AHET & Betreuer: Prof. Dipl.-Ing. Reinhold Benedikter\\
      &       &      \\
\multicolumn{3}{l}{\textbf{Entwicklung der mechanischen Komponenten}} \\
Tobias Schmeisser & 5AHET & Betreuer: Prof. Dipl.-Ing. Peter Lindmoser\\
\end{tabular}
\end{center}
\vspace{\fill}

\begin{minipage}{0.49\textwidth}
Höhere Technische Bundeslehr-\\
und Versuchsanstalt Salzburg \par \medskip 
Itzlinger Hauptstraße 30 \par \medskip 
A-5022 Salzburg \par \medskip 
www.htl-salzburg.ac.at
\end{minipage}
\begin{minipage}{0.49\textwidth} \hspace*{\fill}
{\includegraphics[scale=0.25]{Illustrations/HTL_logo.png}} \hspace{\fill}
\end{minipage}
\newpage
\clearpage
\thispagestyle{empty}
\mbox{}
\newpage
\pagenumbering{Roman}

%% Inhaltsverzeichnis %%%%%%%%%%%%%%%%%%%%%%%%%%%%%%%%%%%%%%%
\tableofcontents %Inhaltsverzeichnis
\bibliographystyle{plain}
\bibliography{literature}

\cleardoublepage %Das erste Kapitel soll auf einer ungeraden Seite beginnen.


\pagestyle{plain} %%Ab hier die Kopf-/Fusszeilen: headings / fancy / ...

%Input files
\pagenumbering{arabic}
\chapter{Einleitung}

Das ist meine Einleitung

\section{Section}
Das ist meine Section.
\newline	%Damit kann man einen Zeilenumbruch erzwingen
Die hat einen Zeilenumbruch.

\subsection{Subsection}
Text in der Subsection
\subsubsection{Subsubsection}
Das ist der Text in der subsubsection. Hier findet man das HTL-Logo (siehe Abbildung \ref{fig:HTL_Logo})

%Hier wird ein Bild eingebunden
\begin{figure}[h]
	\centering
		\includegraphics[scale=0.5]{./Illustrations/HTL_Logo.jpg}
	\caption{Logo der HTL Salzburg}
	\label{fig:HTL_Logo}
\end{figure}

Das ist meine Aufz�hlung:
\begin{itemize}
	\item Erstes Item
	\item Zweites Item
	\item letztes Item
\end{itemize}

\begin{enumerate}
	\item Erstes Item
	\item Zweites Item
	\item letztes Item
\end{enumerate}

\begin{table*}
	\centering
		\begin{tabular}{|c|c|}
	\hline
	     Erste Spalte laaaaaanger Text & Zweite Spalte \\ \hline
			Zweite Zeile & Zweite Zeile 2 \\ \hline
		\end{tabular}
	\caption{Meine Testtabelle}
	\label{tab:MeineTesttabelle}
\end{table*}

Die Spannung betraegt: $\frac{1}{3}\cdot \sqrt{4} \int{6} 6^7 U=3V$ weiter


\paragraph{Paragraph}
Text zum Paragraphen\footnote{Der Text zur Fu�note}, siehe \cite{mann}		%in diesem Kapitel ist Einleitung

%%%%%%%%%%%%%%%%%%%%%%%%%%%%%%%%%%%%%%%%%%%%%%%%%%%%%%%%%%%%%
%% LITERATUR UND ANDERE VERZEICHNISSE
%%%%%%%%%%%%%%%%%%%%%%%%%%%%%%%%%%%%%%%%%%%%%%%%%%%%%%%%%%%%%
%% Ein kleiner Abstand zu den Kapiteln im Inhaltsverzeichnis (toc)
\addtocontents{toc}{\protect\vspace*{\baselineskip}}

%% Literaturverzeichnis
%% ==> Eine Datei 'literatur.bib' wird hierfür benötigt.
%% ==> Sie müssen hierfür BibTeX verwenden (Projekt | Eigenschaften... | BibTeX)
%\addcontentsline{toc}{chapter}{Literaturverzeichnis}
%\nocite{*} %Auch nicht-zitierte BibTeX-Einträge werden angezeigt.
%\bibliographystyle{alpha} %Art der Ausgabe: plain / apalike / amsalpha / ...
%\bibliography{literatur} %Eine Datei 'literatur.bib' wird hierfür benötigt.

%% Abbildungsverzeichnis
%\clearpage
%\addcontentsline{toc}{chapter}{Abbildungsverzeichnis}
%\listoffigures

%% Tabellenverzeichnis
%\clearpage
%\addcontentsline{toc}{chapter}{Tabellenverzeichnis}
%\listoftables


%%%%%%%%%%%%%%%%%%%%%%%%%%%%%%%%%%%%%%%%%%%%%%%%%%%%%%%%%%%%%
%% ANHÄNGE
%%%%%%%%%%%%%%%%%%%%%%%%%%%%%%%%%%%%%%%%%%%%%%%%%%%%%%%%%%%%%
%\appendix
%% ==> Schreiben Sie hier Ihren Text oder fügen Sie externe Dateien ein.

%\input{Dateiname} %Eine Datei 'Dateiname.tex' wird hierfür benötigt.


\end{document}