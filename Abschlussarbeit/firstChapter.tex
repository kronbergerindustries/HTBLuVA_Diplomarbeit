\chapter{Einleitung}

Das ist meine Einleitung

\section{Section}
Das ist meine Section.
\newline	%Damit kann man einen Zeilenumbruch erzwingen
Die hat einen Zeilenumbruch.

\subsection{Subsection}
Text in der Subsection
\subsubsection{Subsubsection}
Das ist der Text in der subsubsection. Hier findet man das HTL-Logo (siehe Abbildung \ref{fig:HTL_Logo})

%Hier wird ein Bild eingebunden
\begin{figure}[h]
	\centering
		\includegraphics[scale=0.5]{./Illustrations/HTL_Logo.jpg}
	\caption{Logo der HTL Salzburg}
	\label{fig:HTL_Logo}
\end{figure}

Das ist meine Aufz�hlung:
\begin{itemize}
	\item Erstes Item
	\item Zweites Item
	\item letztes Item
\end{itemize}

\begin{enumerate}
	\item Erstes Item
	\item Zweites Item
	\item letztes Item
\end{enumerate}

\begin{table*}
	\centering
		\begin{tabular}{|c|c|}
	\hline
	     Erste Spalte laaaaaanger Text & Zweite Spalte \\ \hline
			Zweite Zeile & Zweite Zeile 2 \\ \hline
		\end{tabular}
	\caption{Meine Testtabelle}
	\label{tab:MeineTesttabelle}
\end{table*}

Die Spannung betraegt: $\frac{1}{3}\cdot \sqrt{4} \int{6} 6^7 U=3V$ weiter


\paragraph{Paragraph}
Text zum Paragraphen\footnote{Der Text zur Fu�note}, siehe \cite{mann}